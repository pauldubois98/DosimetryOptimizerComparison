\documentclass[a4paper]{article}

%opening
\title{Radiotherapy Dosimetry: A Review on Open-Source Optimizer}
\author{Paul Dubois}

\begin{document}
	
	\maketitle
	
	\begin{abstract}
		Radiotherapy dosimetry plays a crucial role in optimizing treatment plans for cancer patients.
		In this study, we investigate the performance of a dozen standard state-of-the-art open-source optimizers for radiotherapy dosimetry.
		Our evaluation includes the use of TGG119 benchmark cases as well as one real case obtained from the Institute du Cancer de Montpellier (ICM).
		Among the tested optimizers, Newton CG demonstrates the fastest convergence in terms of the number of iterations.
		However, when considering the computation time per iteration, LBFGS emerges as the most efficient optimizer.
		These findings shed light on the performance of open-source optimizers for radiotherapy dosimetry, aiding practitioners in selecting suitable optimization tools for efficient treatment planning.
	\end{abstract}
	
	\section{Introduction}
	Radiotherapy, a widely utilized intervention for cancer treatment, employs ionizing radiation to eliminate malignant cells.
	Intensity-modulated radiation therapy (IMRT) has emerged as a notable technique within radiotherapy, aiming to deliver high radiation doses to tumors while minimizing exposure to healthy surrounding tissues.
	Traditional IMRT strategies typically employ a set number of beams, often 5, 7, or 9, originating from various angles around the patient, commonly distributed evenly.
	Each beam's intensity is modulated to optimize the delivery of radiation doses to the tumor while reducing exposure to healthy tissues.
	This approach surpasses the effectiveness of the 3D-conformal radiotherapy (3D-CRT) technique.
	To facilitate precise and efficient radiation delivery, a computer-controlled device called the multi-leaf collimator (MLC) is utilized to shape the radiation beam according to the contours of the tumor.
	
	The effectiveness of a radiotherapy treatment plan relies on the optimization procedure, which involves a series of steps aimed at ensuring the optimal delivery of radiation in accordance with the prescribed guidelines of medical practitioners.
	Typically, computer software is employed to facilitate the optimization process, taking into account various factors such as patient anatomy, the size and location of the tumor and organs, and the radiation objectives defined by medical professionals.
	
	\paragraph{Pre-dose-optimization}
	\paragraph{Radiotherapy doses}
	\paragraph{Dose-Volume Histograms}
	\paragraph{Dose Evaluation}
	
	\section{Methods}
	
	\section{Results}
	
	\section{Discussion}
	
	
\end{document}
